\documentclass{report}

\usepackage[warn]{mathtext}
\usepackage[T2A]{fontenc}
\usepackage[utf8]{inputenc}
\usepackage[english, russian]{babel}
\usepackage[pdftex]{hyperref}
\usepackage{tempora}
\usepackage[12pt]{extsizes}
\usepackage{listings}
\usepackage{color}
\usepackage{geometry}
\usepackage{enumitem}
\usepackage{multirow}
\usepackage{graphicx}
\usepackage{indentfirst}
\usepackage{amsmath}

\geometry{a4paper,top=2cm,bottom=2cm,left=2.5cm,right=1.5cm}
\setlength{\parskip}{0.5cm}
\setlist{nolistsep, itemsep=0.3cm,parsep=0pt}

\usepackage{listings}
\lstset{language=C++,
        basicstyle=\footnotesize,
		keywordstyle=\color{blue}\ttfamily,
		stringstyle=\color{red}\ttfamily,
		commentstyle=\color{green}\ttfamily,
		morecomment=[l][\color{red}]{\#}, 
		tabsize=4,
		breaklines=true,
  		breakatwhitespace=true,
  		title=\lstname,       
}

\makeatletter
\renewcommand\@biblabel[1]{#1.\hfil}
\makeatother

\begin{document}

\begin{titlepage}

\begin{center}
Министерство науки и высшего образования Российской Федерации
\end{center}

\begin{center}
Федеральное государственное автономное образовательное учреждение высшего образования \\
Национальный исследовательский Нижегородский государственный университет им. Н.И. Лобачевского
\end{center}

\begin{center}
Институт информационных технологий, математики и механики
\end{center}

\vspace{4em}

\begin{center}
\textbf{\LargeОтчет по лабораторной работе} \\
\end{center}
\begin{center}
\textbf{\Large«Вычисление многомерных интегралов с использованием многошаговой схемы (метод Симпсона)»} \\
\end{center}

\vspace{4em}

\newbox{\lbox}
\savebox{\lbox}{\hbox{text}}
\newlength{\maxl}
\setlength{\maxl}{\wd\lbox}
\hfill\parbox{7cm}{
\hspace*{5cm}\hspace*{-5cm}\textbf{Выполнил:} \\ студент группы 381906-3 \\ Зарубин М.П.\\
\\
\hspace*{5cm}\hspace*{-5cm}\textbf{Проверил:}\\ доцент кафедры МОСТ, \\ кандидат технических наук \\ Сысоев А. В.\\
}
\vspace{\fill}

\begin{center} Нижний Новгород \\ 2022 \end{center}

\end{titlepage}

\setcounter{page}{2}

% Содержание
\tableofcontents
\newpage

% Введение
\section*{Введение}
\addcontentsline{toc}{section}{Введение}
\par Вычисление многомерных интегралов - математически сложная, но регулярно используемая во многих проектах задача, поэтому на сегодняшний день существуют различные алгоритмы ее решения, которые можно подразделить на точные и приближенные. Методы, вычисляющие точное значение интеграла, сложны в реализации и работают сравниетльно долго, а вот варианты, считающие результат приближенно, прекрасно подходят для большинства компьютерных программ, так как их ошибка достаточно мала и компенсируется высокой скорстью вычислений. Метод Симпсона - один из таких алгоритмов, приближающий подынтегральную функцию некоторым количеством парабол.
\newpage

% Постановка задачи
\section*{Постановка задачи}
\addcontentsline{toc}{section}{Постановка задачи}
\par В данной лабораторной работе требуется реализовать последовательную и параллельные версии алгоримта вычисления многомерных интегралов методом Симпсона, используя технологии OpenMP, TBB и std::thread. Для проверки корректности выполненной задачи необходимо провести тестирование написанной программы посредством фреймворка Google Test. 
\par Также нужно сделать выводы об эффективности распараллеливания с помощью той или иной технологии, замерив и сравнив время работы алгоритма на одинаковых данных.
\newpage

% Описание алгоритма
\section*{Описание алгоритма}
\addcontentsline{toc}{section}{Описание алгоритма}
\par Вычисление интегралов с использованием многошаговой схемы подразумевает приближение подинтегральной функции с помощью более простых функций - в методе Симпсона используются параболы. Точность вычисления зависит от количества разбиений, то есть количества приближающих подинтегральную функцию парабол, поэтому для начала мы задаем это число, а после считаем значение интеграла на каждом отрезке. 
\par Для одномерного интеграла метод Симпсона описывает данный процесс следующим выражением: $$ \frac{(b - a)} {6 * k} * (f_{a} + 4 * f_{m} + f_{b})$$где $b$ и $а$ верхняя и нижняя границы интегрируемого диапазона, $k$ - число разбиений, а $f_{a}$, $f_{m}$ и $f_b$ значения функции в левой, средней и правой точках интервала.
\par В случае, если стоит задача вычислить многомерный интеграл, нам необходимо применить формулу на каждой размерности и перемножить полученные результаты. 
\newpage

% Описание схемы распараллеливания
\section*{Описание схемы распараллеливания}
\addcontentsline{toc}{section}{Описание схемы распараллеливания}
\par Распараллеливание программы происходит с помощью распределения данных и вычислительных операций между потоками. То есть мы разбиваем основной и наиболее трудоимкий цикл, реализующий формулу Симпсона, на несколько циклов с меньшим количеством итераций, которые выполняются независимо на разных потоках.
\par В OpenMP распределение происходит автоматически, мы лишь указываем приватные (разделяемые между потоками) и локальные (не разделяемые между потоками) переменные через директиву \#pragma omp parallel. Для суммирования вычисленных значений и получения конечного результата в главном потоке применяется редукция.
\par В TBB распределение также происходит автоматически, и в случае использования функционального подхода с tbb::parallel\_reduce, который оказался наиболее подходящим для моей задачи, требуется указать только диапазон blocked\_range, начальное значение редуцируемой переменной, функцию в виде лямбда-выражения и операцию, которую будет использовать редукция для формирования конечного результата.
\par Распрараллеливание с использованием std::thread требует больше действий от программиста, предоставляя лишь сущность потока (thread), куда мы можем передать функцию и необходимые ей параметры. Вычисление количества операций на поток производим вручную, то есть вычисляем стартовое и конечное значение переменной цикла в зависимости от номера потока. Суммирование полученных результатов также делаем сами - для этого я передавал в функцию указатель на i-ый элемент вектора, где i - номер потока, после записывал в эту память результат, а по завершению работы потока уже в основной функции суммировал полученное в каждом потоке значение в единую переменную. 
\newpage

% Описание программной реализации
\section*{Описание программной реализации}
\addcontentsline{toc}{section}{Описание программной реализации}
Программа состоит из заголовочного файла simpson\_method.h и двух файлов исходного кода simpson\_method.cpp и main.cpp.
\par В заголовочном файле находятся прототипы функций для последовательного и параллельных алгоритмов вычисления интеграла методом Симпсона, в simpson\_method.cpp - их реализация, а файл main.cpp содержит тесты, проверяющие корректность и эффективность работы программы.
\par Функция для последовательного алгоритма:
\begin{lstlisting}
double simpsonMethodSeq(sizeType dimension,
    std::vector<double> leftBorders, std::vector<double> rightBorders,
    std::function<double(std::vector<double>)> function, std::vector<sizeType> countParts);
\end{lstlisting}
\par Функция для параллельного алгоритма:
\begin{lstlisting}
double simpsonMethodParallel(sizeType dimension,
    std::vector<double> leftBorders, std::vector<double> rightBorders,
    std::function<double(std::vector<double>)> function, std::vector<sizeType> countParts,
    int threadCount);
\end{lstlisting}
\par Параметр dimension соответствует размерности функции, вектора leftBorders и rightBorders содержат нижние и верхние границы интеграла по каждой из переменной, function - исходная функция, а в векторе countParts записано число разбиений по размерностям. Параллельная функция также принимает параметр threadCount, куда мы можем передать желаемое число задействованных потоков.
\newpage

% Подтверждение корректности
\section*{Подтверждение корректности}
\addcontentsline{toc}{section}{Подтверждение корректности}
Для подтверждения корректности реализованного метода Симпсона я написал 6 тестов для каждой из версий алгоритма (последовательной и параллельных). Один из них проверяет, что функция не выбрасывает исключений во время работы, в остальных сравнивается вычисленное программой значение интеграла с правильным ответом. Тестирование происходит на функциях от 1, 2, 3, 4 и 5 переменных.
\par Также предусмотрен тест на эффективность, вычисляющий ускорение параллельной версии относительно последовательной.
\newpage

% Результаты экспериментов
\section*{Результаты экспериментов}
\addcontentsline{toc}{section}{Результаты экспериментов}
Вычислительные эксперименты для оценки эффективности работы параллельных алгоритмов проводились на ноутбуке со следующими характеристиками:
\begin{itemize}
\item Процессор: AMD Ryzen 5 5500U, 2.1 ГГц (4.0 ГГц, в режиме Turbo), количество ядер: 6 физических, 12 логических;
\item Оперативная память: 16 ГБ (DDR4), 3200 МГц;
\item Операционная система: Windows 10 Home.
\end{itemize}

\par Эксперименты проводились на 12 потоках и функции 5 переменных с разбиением 7 по каждой размерности. 

\par Результаты экспериментов представлены в Таблице.
\begin{table}[!h]
\caption{Результаты вычислительных экспериментов}
\centering
\begin{tabular}{| p{2cm} | p{3cm} | p{4cm} | p{2cm} |}
\hline
Версия алгоритма & Время работы (в секундах) & Ускорение  \\[5pt]
\hline
Sequential        & 13.878527        & -         \\
OpenMP        & 2.381890        & 5.826687         \\
TBB       & 1.537655        & 9.025774      \\
std::thread        &  1.998000        & 6.946210           \\

\hline
\end{tabular}
\end{table}

\newpage

% Выводы из результатов экспериментов
\section*{Выводы из результатов экспериментов}
\addcontentsline{toc}{section}{Выводы из результатов экспериментов}
Из данных, полученных в результате экспериментов, можно сделать вывод, что параллельные версии работают намного быстрее, чем последовательная. Во всех случаях удалось достичь ускорения более, чем в 5 раз, а наилучшие результаты показал TBB, улучшив производительность программы в 9 раз. Также стоит отметить, что ускорение оказалось меньше числа задействованных потоков, так как помимо нашей программы в системе работают и другие, конкурируя за ресурсы процессора. 
\newpage

% Заключение
\section*{Заключение}
\addcontentsline{toc}{section}{Заключение}
В рамках лабораторной работы были реализованы последовательный и параллельный алгоритмы вычисления многомерных интегралов с использованием многошаговой схемы (метод Симпсона). Проведенные тесты показали корректность написанного кода, а замеры времени - эффективность параллельных версий.
\newpage

% Литература
\section*{Литература}
\addcontentsline{toc}{section}{Литература}
\begin{enumerate}
\item Метод Симпсона - Электронный ресурс. URL: \newline \url{https://en.wikipedia.org/wiki/Simpson%27s_rule}
\item OpenMP - Электронный ресурс. URL: \newline \url{https://www.openmp.org/wp-content/uploads/OpenMP-4.0-C.pdf}
\item TBB - Электронный ресурс. URL: \newline \url{https://habr.com/ru/post/102670/}
\item std::thread - Электронный ресурс. URL: \newline \url{https://en.cppreference.com/w/cpp/thread/thread}
\item А.В. Сысоев, И.Б. Мееров, А.А. Сиднев «Средства разработки параллельных программ для систем с общей памятью. Библиотека Intel Threading Building Blocks». Нижний Новгород, 2007, 128 с.
\end{enumerate} 
\newpage

% Приложение
\section*{Приложение}
\addcontentsline{toc}{section}{Приложение}
\textbf{Последовательная версия}
\newline
\newline simpson\_method.h
\begin{lstlisting}
// Copyright 2022 Zarubin Mikhail

#ifndef MODULES_TASK_1_ZARUBIN_M_SIMPSON_METHOD_SIMPSON_METHOD_H_
#define MODULES_TASK_1_ZARUBIN_M_SIMPSON_METHOD_SIMPSON_METHOD_H_

#include<functional>
#include<vector>

const double ERROR = 0.001;

using sizeType = std::vector<double>::size_type;


double simpsonMethod(sizeType dimension,
    std::vector<double> leftBorders, std::vector<double> rightBorders,
    std::function<double(std::vector<double>)> function, std::vector<sizeType> countParts);

#endif  // MODULES_TASK_1_ZARUBIN_M_SIMPSON_METHOD_SIMPSON_METHOD_H_

\end{lstlisting}
simpson\_method.cpp
\begin{lstlisting}
// Copyright 2022 Zarubin Mikhail

#include <random>
#include "../../../modules/task_1/zarubin_m_simpson_method/simpson_method.h"

struct Partition {
    double left;
    double center;
    double right;

    Partition() = default;
    Partition(double _left, double _center, double _right) :
        left(_left),
        center(_center),
        right(_right)
    {};
};

double simpsonMethod(sizeType dimension,
    std::vector<double> leftBorders, std::vector<double> rightBorders,
    std::function<double(std::vector<double>)> function, std::vector<sizeType> countParts) {
    std::vector<double> coeffs(dimension);
    sizeType iterCount = 1;

    for (sizeType i = 0; i < dimension; i++) {
        coeffs[i] = (rightBorders[i] - leftBorders[i]) / countParts[i];
        iterCount *= countParts[i];
    }

    double integralValue = 0.0;
    sizeType count = static_cast<sizeType>(std::pow(6, dimension));
    std::vector<Partition> partitions(dimension);
    std::vector<double> args(dimension);

    for (sizeType i = 0; i < iterCount; i++) {
        sizeType temp = i;
        for (sizeType j = 0; j < dimension; j++) {
            double left = leftBorders[j] + temp % countParts[j] * coeffs[j];
            double right = leftBorders[j] + temp % countParts[j] * coeffs[j] + coeffs[j];
            double center = (right + left) / 2;

            partitions[j] = Partition(left, center, right);

            temp /= countParts[j];
        }

        for (sizeType k = 0; k < count; k++) {
            temp = k;

            for (sizeType j = 0; j < dimension; j++) {
                switch (temp % 6) {
                case 1:
                    args[j] = partitions[j].right; break;
                case 5:
                    args[j] = partitions[j].left; break;
                default:
                    args[j] = partitions[j].center; break;
                }

                temp /= 6;
            }

            integralValue += function(args);
        }
    }

    for (sizeType i = 0; i < dimension; i++) {
        integralValue *= (coeffs[i] / 6);
    }

    return integralValue;
}

\end{lstlisting}
main.cpp
\begin{lstlisting}
// Copyright 2022 Zarubin Mikhail

#include <gtest/gtest.h>
#include "./simpson_method.h"


TEST(SimpsonMethod, can_run_simpson_method) {
    std::vector<double> leftsBorders = { 1, 2 };
    std::vector<double> rightBorders = { 11, 9 };
    std::function<double(std::vector<double>)> func =
        [](std::vector<double> args) { return args[0] * args[1]; };
    sizeType dimension = 2;
    std::vector<sizeType> countParts = { 8, 4 };

    ASSERT_NO_THROW(simpsonMethod(dimension, leftsBorders, rightBorders,
        func, countParts));
}

TEST(SimpsonMethod, simpos_method_works_correctly_with_one_variable) {
    std::vector<double> leftsBorders = { 2 };
    std::vector<double> rightBorders = { 4 };
    std::function<double(std::vector<double>)> funcOneVariable =
        [](std::vector<double> args) { return args[0]; };
    sizeType dimension = 1;
    std::vector<sizeType> countParts = { 10 };

    double expectedResult = 6.0;
    double result = simpsonMethod(dimension, leftsBorders, rightBorders,
        funcOneVariable, countParts);

    ASSERT_NEAR(expectedResult, result, ERROR);
}

TEST(SimpsonMethod, simpos_method_works_correctly_with_two_variable) {
    std::vector<double> leftsBorders = { 2, -1};
    std::vector<double> rightBorders = { 4, 5 };
    std::function<double(std::vector<double>)> funcTwoVariable =
        [](std::vector<double> args) { return args[0] * args[1]; };
    sizeType dimension = 2;
    std::vector<sizeType> countParts = { 5, 5 };

    double expectedResult = 72.0;
    double result = simpsonMethod(dimension, leftsBorders, rightBorders,
        funcTwoVariable, countParts);

    ASSERT_NEAR(expectedResult, result, ERROR);
}

TEST(SimpsonMethod, simpos_method_works_correctly_with_three_variable) {
    std::vector<double> leftsBorders = { 2, -1, 0 };
    std::vector<double> rightBorders = { 4, 5, 3 };
    std::function<double(std::vector<double>)> funcThreeVariable =
        [](std::vector<double> args) { return args[0] * args[1] * args[2]; };
    sizeType dimension = 3;
    std::vector<sizeType> countParts = { 4, 4, 4 };

    double expectedResult = 324.0;
    double result = simpsonMethod(dimension, leftsBorders, rightBorders,
        funcThreeVariable, countParts);

    ASSERT_NEAR(expectedResult, result, ERROR);
}

TEST(SimpsonMethod, simpos_method_works_correctly_with_four_variable) {
    std::vector<double> leftsBorders = { 2, -1, 0, -3 };
    std::vector<double> rightBorders = { 4, 5, 3, 1 };
    std::function<double(std::vector<double>)> funcFourVariable =
        [](std::vector<double> args) { return args[0] * args[1] * args[2] * args[3]; };
    sizeType dimension = 4;
    std::vector<sizeType> countParts = { 3, 3, 3, 3 };

    double expectedResult = -1296.0;
    double result = simpsonMethod(dimension, leftsBorders, rightBorders,
        funcFourVariable, countParts);

    ASSERT_NEAR(expectedResult, result, ERROR);
}

TEST(SimpsonMethod, simpos_method_works_correctly_with_five_variable) {
    std::vector<double> leftsBorders = { 2, -1, 0, -3, 1 };
    std::vector<double> rightBorders = { 4, 5, 3, 1, 2 };
    std::function<double(std::vector<double>)> funcFiveVariable =
        [](std::vector<double> args) { return args[0] * args[1] * args[2] * args[3] * args[4]; };
    sizeType dimension = 5;
    std::vector<sizeType> countParts = { 2, 2, 2, 2, 2 };

    double expectedResult = -1944.0;
    double result = simpsonMethod(dimension, leftsBorders, rightBorders,
        funcFiveVariable, countParts);

    ASSERT_NEAR(expectedResult, result, ERROR);
}

\end{lstlisting}

\textbf{OpenMP version}
\newline
\newline simpson\_method.h
\begin{lstlisting}
// Copyright 2022 Zarubin Mikhail

#ifndef MODULES_TASK_2_ZARUBIN_M_SIMPSON_METHOD_SIMPSON_METHOD_H_
#define MODULES_TASK_2_ZARUBIN_M_SIMPSON_METHOD_SIMPSON_METHOD_H_

#include <functional>
#include <vector>

const double ERROR = 0.001;

using sizeType = std::vector<double>::size_type;


double simpsonMethodSeq(sizeType dimension,
    std::vector<double> leftBorders, std::vector<double> rightBorders,
    std::function<double(std::vector<double>)> function, std::vector<sizeType> countParts);

double simpsonMethodParallel(sizeType dimension,
    std::vector<double> leftBorders, std::vector<double> rightBorders,
    std::function<double(std::vector<double>)> function, std::vector<sizeType> countParts);

#endif  // MODULES_TASK_2_ZARUBIN_M_SIMPSON_METHOD_SIMPSON_METHOD_H_

\end{lstlisting}
simpson\_method.cpp
\begin{lstlisting}
// Copyright 2022 Zarubin Mikhail

#include <omp.h>
#include <random>
#include "../../../modules/task_2/zarubin_m_simpson_method/simpson_method.h"

struct Partition {
    double left;
    double center;
    double right;

    Partition() = default;
    Partition(double _left, double _center, double _right) :
        left(_left),
        center(_center),
        right(_right)
    {};
};

double simpsonMethodSeq(sizeType dimension,
    std::vector<double> leftBorders, std::vector<double> rightBorders,
    std::function<double(std::vector<double>)> function, std::vector<sizeType> countParts) {
    std::vector<double> coeffs(dimension);
    sizeType iterCount = 1;

    for (sizeType i = 0; i < dimension; i++) {
        coeffs[i] = (rightBorders[i] - leftBorders[i]) / countParts[i];
        iterCount *= countParts[i];
    }

    double integralValue = 0.0;
    sizeType count = static_cast<sizeType>(std::pow(6, dimension));
    std::vector<Partition> partitions(dimension);
    std::vector<double> args(dimension);

    for (sizeType i = 0; i < iterCount; i++) {
        sizeType temp = i;
        for (sizeType j = 0; j < dimension; j++) {
            double left = leftBorders[j] + temp % countParts[j] * coeffs[j];
            double right = leftBorders[j] + temp % countParts[j] * coeffs[j] + coeffs[j];
            double center = (right + left) / 2;

            partitions[j] = Partition(left, center, right);

            temp /= countParts[j];
        }

        for (sizeType k = 0; k < count; k++) {
            temp = k;

            for (sizeType j = 0; j < dimension; j++) {
                switch (temp % 6) {
                case 1:
                    args[j] = partitions[j].right; break;
                case 5:
                    args[j] = partitions[j].left; break;
                default:
                    args[j] = partitions[j].center; break;
                }

                temp /= 6;
            }

            integralValue += function(args);
        }
    }

    for (sizeType i = 0; i < dimension; i++) {
        integralValue *= (coeffs[i] / 6);
    }

    return integralValue;
}

double simpsonMethodParallel(sizeType dimension,
    std::vector<double> leftBorders, std::vector<double> rightBorders,
    std::function<double(std::vector<double>)> function, std::vector<sizeType> countParts) {
    std::vector<double> coeffs(dimension);
    sizeType iterCount = 1;

    for (sizeType i = 0; i < dimension; i++) {
        coeffs[i] = (rightBorders[i] - leftBorders[i]) / countParts[i];
        iterCount *= countParts[i];
    }

    double integralValue = 0.0;
    sizeType count = static_cast<sizeType>(std::pow(6, dimension));
    std::vector<Partition> partitions(dimension);
    std::vector<double> args(dimension);

#pragma omp parallel shared(dimension, iterCount, coeffs, leftBorders, count, countParts) \
  firstprivate(partitions, args) reduction(+ : integralValue)
#pragma omp for schedule(static)
    for (int i = 0; i < static_cast<int>(iterCount); i++) {
        sizeType temp = i;
        for (sizeType j = 0; j < dimension; j++) {
            double left = leftBorders[j] + temp % countParts[j] * coeffs[j];
            double right = leftBorders[j] + temp % countParts[j] * coeffs[j] + coeffs[j];
            double center = (right + left) / 2;

            partitions[j] = Partition(left, center, right);

            temp /= countParts[j];
        }

        for (sizeType k = 0; k < count; k++) {
            temp = k;

            for (sizeType j = 0; j < dimension; j++) {
                switch (temp % 6) {
                case 1:
                    args[j] = partitions[j].right; break;
                case 5:
                    args[j] = partitions[j].left; break;
                default:
                    args[j] = partitions[j].center; break;
                }

                temp /= 6;
            }

            integralValue += function(args);
        }
    }

#pragma omp for schedule(static)
    for (int i = 0; i < static_cast<int>(dimension); i++) {
        integralValue *= (coeffs[i] / 6);
    }

    return integralValue;
}

\end{lstlisting}
main.cpp
\begin{lstlisting}
// Copyright 2022 Zarubin Mikhail

#include <gtest/gtest.h>
#include <omp.h>
#include "./simpson_method.h"


TEST(SimpsonMethod, can_run_simpson_method_parallel) {
    std::vector<double> leftsBorders = { 1, 2 };
    std::vector<double> rightBorders = { 11, 9 };
    std::function<double(std::vector<double>)> func =
        [](std::vector<double> args) { return args[0] * args[1]; };
    sizeType dimension = 2;
    std::vector<sizeType> countParts = { 8, 4 };

    ASSERT_NO_THROW(simpsonMethodParallel(dimension, leftsBorders, rightBorders,
        func, countParts));
}

TEST(SimpsonMethod, simpos_method_parallel_works_correctly_with_one_variable) {
    std::vector<double> leftsBorders = { 2 };
    std::vector<double> rightBorders = { 4 };
    std::function<double(std::vector<double>)> funcOneVariable =
        [](std::vector<double> args) { return args[0]; };
    sizeType dimension = 1;
    std::vector<sizeType> countParts = { 10 };

    double expectedResult = 6.0;
    double result = simpsonMethodParallel(dimension, leftsBorders, rightBorders,
        funcOneVariable, countParts);

    ASSERT_NEAR(expectedResult, result, ERROR);
}

TEST(SimpsonMethod, simpos_method_parallel_works_correctly_with_two_variable) {
    std::vector<double> leftsBorders = { 2, -1 };
    std::vector<double> rightBorders = { 4, 5 };
    std::function<double(std::vector<double>)> funcTwoVariable =
        [](std::vector<double> args) { return args[0] * args[1]; };
    sizeType dimension = 2;
    std::vector<sizeType> countParts = { 5, 5 };

    double expectedResult = 72.0;
    double result = simpsonMethodParallel(dimension, leftsBorders, rightBorders,
        funcTwoVariable, countParts);

    ASSERT_NEAR(expectedResult, result, ERROR);
}

TEST(SimpsonMethod, simpos_method_parallel_works_correctly_with_three_variable) {
    std::vector<double> leftsBorders = { 2, -1, 0 };
    std::vector<double> rightBorders = { 4, 5, 3 };
    std::function<double(std::vector<double>)> funcThreeVariable =
        [](std::vector<double> args) { return args[0] * args[1] * args[2]; };
    sizeType dimension = 3;
    std::vector<sizeType> countParts = { 4, 4, 4 };

    double expectedResult = 324.0;
    double result = simpsonMethodParallel(dimension, leftsBorders, rightBorders,
        funcThreeVariable, countParts);

    ASSERT_NEAR(expectedResult, result, ERROR);
}

TEST(SimpsonMethod, simpos_method_parallel_works_correctly_with_four_variable) {
    std::vector<double> leftsBorders = { 2, -1, 0, -3 };
    std::vector<double> rightBorders = { 4, 5, 3, 1 };
    std::function<double(std::vector<double>)> funcFourVariable =
        [](std::vector<double> args) { return args[0] * args[1] * args[2] * args[3]; };
    sizeType dimension = 4;
    std::vector<sizeType> countParts = { 3, 3, 3, 3 };

    double expectedResult = -1296.0;
    double result = simpsonMethodParallel(dimension, leftsBorders, rightBorders,
        funcFourVariable, countParts);

    ASSERT_NEAR(expectedResult, result, ERROR);
}

TEST(SimpsonMethod, simpos_method_parallel_works_correctly_with_five_variable) {
    std::vector<double> leftsBorders = { 2, -1, 0, -3, 1 };
    std::vector<double> rightBorders = { 4, 5, 3, 1, 2 };
    std::function<double(std::vector<double>)> funcFiveVariable =
        [](std::vector<double> args) { return args[0] * args[1] * args[2] * args[3] * args[4]; };
    sizeType dimension = 5;
    std::vector<sizeType> countParts = { 2, 2, 2, 2, 2 };

    double expectedResult = -1944.0;
    double result = simpsonMethodParallel(dimension, leftsBorders, rightBorders,
        funcFiveVariable, countParts);

    ASSERT_NEAR(expectedResult, result, ERROR);
}

TEST(DISABLED_SimpsonMethod, boost_test) {
    std::vector<double> leftsBorders = { 2, -1, 0, -3, 1 };
    std::vector<double> rightBorders = { 4, 5, 3, 1, 2 };
    std::function<double(std::vector<double>)> funcFiveVariable =
        [](std::vector<double> args) { return args[0] * args[1] * args[2] * args[3] * args[4]; };
    sizeType dimension = 5;
    std::vector<sizeType> countParts(5, 6);

    auto startSeq = omp_get_wtime();
    simpsonMethodSeq(dimension, leftsBorders, rightBorders, funcFiveVariable, countParts);
    auto finishSeq = omp_get_wtime();

    omp_set_num_threads(6);
    auto startParallel = omp_get_wtime();
    simpsonMethodParallel(dimension, leftsBorders, rightBorders, funcFiveVariable, countParts);
    auto finishParallel = omp_get_wtime();

    auto seqTime = finishSeq - startSeq;
    auto parallelTime = finishParallel - startParallel;
    auto boost = seqTime / parallelTime;

    printf("seqTime = %lf\nparallelTime = %lf\nboost = %lf\n", seqTime, parallelTime, boost);
}

\end{lstlisting}

\textbf{TBB version}
\newline
\newline simpson\_method.h
\begin{lstlisting}
// Copyright 2022 Zarubin Mikhail

#ifndef MODULES_TASK_3_ZARUBIN_M_SIMPSON_METHOD_SIMPSON_METHOD_H_
#define MODULES_TASK_3_ZARUBIN_M_SIMPSON_METHOD_SIMPSON_METHOD_H_

#include <tbb/task_scheduler_init.h>
#include <functional>
#include <vector>

using sizeType = std::vector<double>::size_type;


double simpsonMethodSeq(sizeType dimension,
    std::vector<double> leftBorders, std::vector<double> rightBorders,
    std::function<double(std::vector<double>)> function, std::vector<sizeType> countParts);

double simpsonMethodParallel(sizeType dimension,
    std::vector<double> leftBorders, std::vector<double> rightBorders,
    std::function<double(std::vector<double>)> function, std::vector<sizeType> countParts,
    int processCount = tbb::task_scheduler_init::default_num_threads());

#endif  // MODULES_TASK_3_ZARUBIN_M_SIMPSON_METHOD_SIMPSON_METHOD_H_

\end{lstlisting}
simpson\_method.cpp
\begin{lstlisting}
// Copyright 2022 Zarubin Mikhail

#include <tbb/blocked_range.h>
#include <tbb/parallel_reduce.h>
#include <random>
#include "../../../modules/task_3/zarubin_m_simpson_method/simpson_method.h"


struct Partition {
    double left;
    double center;
    double right;

    Partition() = default;
    Partition(double _left, double _center, double _right) :
        left(_left),
        center(_center),
        right(_right)
    {};
};

double simpsonMethodSeq(sizeType dimension,
    std::vector<double> leftBorders, std::vector<double> rightBorders,
    std::function<double(std::vector<double>)> function, std::vector<sizeType> countParts) {
    std::vector<double> coeffs(dimension);
    sizeType iterCount = 1;

    for (sizeType i = 0; i < dimension; i++) {
        coeffs[i] = (rightBorders[i] - leftBorders[i]) / countParts[i];
        iterCount *= countParts[i];
    }

    double integralValue = 0.0;
    sizeType count = static_cast<sizeType>(std::pow(6, dimension));
    std::vector<Partition> partitions(dimension);
    std::vector<double> args(dimension);

    for (sizeType i = 0; i < iterCount; i++) {
        sizeType temp = i;
        for (sizeType j = 0; j < dimension; j++) {
            double left = leftBorders[j] + temp % countParts[j] * coeffs[j];
            double right = leftBorders[j] + temp % countParts[j] * coeffs[j] + coeffs[j];
            double center = (right + left) / 2;

            partitions[j] = Partition(left, center, right);

            temp /= countParts[j];
        }

        for (sizeType k = 0; k < count; k++) {
            temp = k;

            for (sizeType j = 0; j < dimension; j++) {
                switch (temp % 6) {
                case 1:
                    args[j] = partitions[j].right; break;
                case 5:
                    args[j] = partitions[j].left; break;
                default:
                    args[j] = partitions[j].center; break;
                }

                temp /= 6;
            }

            integralValue += function(args);
        }
    }

    for (sizeType i = 0; i < dimension; i++) {
        integralValue *= (coeffs[i] / 6);
    }

    return integralValue;
}

double simpsonMethodParallel(sizeType dimension,
    std::vector<double> leftBorders, std::vector<double> rightBorders,
    std::function<double(std::vector<double>)> function, std::vector<sizeType> countParts,
    int processCount) {
    std::vector<double> coeffs(dimension);
    sizeType iterCount = 1;
    double multiplier = 1;

    for (sizeType i = 0; i < dimension; i++) {
        coeffs[i] = (rightBorders[i] - leftBorders[i]) / countParts[i];
        iterCount *= countParts[i];
        multiplier *= coeffs[i] / 6.0;
    }

    tbb::task_scheduler_init init(processCount);
    double result = tbb::parallel_reduce(
        tbb::blocked_range<sizeType>(0, iterCount), 0.0,
        [&](tbb::blocked_range<sizeType> r, double integralValue) {
            sizeType count = static_cast<sizeType>(std::pow(6, dimension));
            std::vector<Partition> partitions(dimension);
            std::vector<double> args(dimension);

            for (sizeType i = r.begin(); i != r.end(); i++) {
                sizeType temp = i;
                for (sizeType j = 0; j < dimension; j++) {
                    double left = leftBorders[j] + temp % countParts[j] * coeffs[j];
                    double right = leftBorders[j] + temp % countParts[j] * coeffs[j] + coeffs[j];
                    double center = (right + left) / 2;

                    partitions[j] = Partition(left, center, right);

                    temp /= countParts[j];
                }

                for (sizeType k = 0; k < count; k++) {
                    temp = k;

                    for (sizeType j = 0; j < dimension; j++) {
                        switch (temp % 6) {
                        case 1:
                            args[j] = partitions[j].right; break;
                        case 5:
                            args[j] = partitions[j].left; break;
                        default:
                            args[j] = partitions[j].center; break;
                        }

                        temp /= 6;
                    }

                    integralValue += multiplier * function(args);
                }
            }
            return integralValue;
        },
        std::plus<double>());
    return result;
}

\end{lstlisting}
main.cpp
\begin{lstlisting}
// Copyright 2022 Zarubin Mikhail

#include <gtest/gtest.h>
#include <tbb/tick_count.h>
#include "./simpson_method.h"


TEST(SimpsonMethod, can_run_simpson_method_parallel) {
    std::vector<double> leftsBorders = { 1, 2 };
    std::vector<double> rightBorders = { 11, 9 };
    std::function<double(std::vector<double>)> func =
        [](std::vector<double> args) { return args[0] * args[1]; };
    sizeType dimension = 2;
    std::vector<sizeType> countParts = { 8, 4 };

    ASSERT_NO_THROW(simpsonMethodParallel(dimension, leftsBorders, rightBorders,
        func, countParts));
}

TEST(SimpsonMethod, simpos_method_parallel_works_correctly_with_one_variable) {
    std::vector<double> leftsBorders = { 2 };
    std::vector<double> rightBorders = { 4 };
    std::function<double(std::vector<double>)> funcOneVariable =
        [](std::vector<double> args) { return args[0]; };
    sizeType dimension = 1;
    std::vector<sizeType> countParts(1, 10);

    double expectedResult = 6.0;
    double result = simpsonMethodParallel(dimension, leftsBorders, rightBorders,
        funcOneVariable, countParts);

    ASSERT_NEAR(expectedResult, result, 0.001);
}

TEST(SimpsonMethod, simpos_method_parallel_works_correctly_with_two_variable) {
    std::vector<double> leftsBorders = { 2, -1 };
    std::vector<double> rightBorders = { 4, 5 };
    std::function<double(std::vector<double>)> funcTwoVariable =
        [](std::vector<double> args) { return args[0] * args[1]; };
    sizeType dimension = 2;
    std::vector<sizeType> countParts(2, 4);

    double expectedResult = 72.0;
    double result = simpsonMethodParallel(dimension, leftsBorders, rightBorders,
        funcTwoVariable, countParts);

    ASSERT_NEAR(expectedResult, result, 0.001);
}

TEST(SimpsonMethod, simpos_method_parallel_works_correctly_with_three_variable) {
    std::vector<double> leftsBorders = { 2, -1, 0 };
    std::vector<double> rightBorders = { 4, 5, 3 };
    std::function<double(std::vector<double>)> funcThreeVariable =
        [](std::vector<double> args) { return args[0] * args[1] * args[2]; };
    sizeType dimension = 3;
    std::vector<sizeType> countParts(3, 3);

    double expectedResult = 324.0;
    double result = simpsonMethodParallel(dimension, leftsBorders, rightBorders,
        funcThreeVariable, countParts);

    ASSERT_NEAR(expectedResult, result, 0.001);
}

TEST(SimpsonMethod, simpos_method_parallel_works_correctly_with_four_variable) {
    std::vector<double> leftsBorders = { 2, -1, 0, -3 };
    std::vector<double> rightBorders = { 4, 5, 3, 1 };
    std::function<double(std::vector<double>)> funcFourVariable =
        [](std::vector<double> args) { return args[0] * args[1] * args[2] * args[3]; };
    sizeType dimension = 4;
    std::vector<sizeType> countParts(4, 2);

    double expectedResult = -1296.0;
    double result = simpsonMethodParallel(dimension, leftsBorders, rightBorders,
        funcFourVariable, countParts);

    ASSERT_NEAR(expectedResult, result, 0.001);
}

TEST(SimpsonMethod, simpos_method_parallel_works_correctly_with_five_variable) {
    std::vector<double> leftsBorders = { 2, -1, 0, -3, 1 };
    std::vector<double> rightBorders = { 4, 5, 3, 1, 2 };
    std::function<double(std::vector<double>)> funcFiveVariable =
        [](std::vector<double> args) { return args[0] * args[1] * args[2] * args[3] * args[4]; };
    sizeType dimension = 5;
    std::vector<sizeType> countParts(5, 2);

    double expectedResult = -1944.0;
    double result = simpsonMethodParallel(dimension, leftsBorders, rightBorders,
        funcFiveVariable, countParts);

    ASSERT_NEAR(expectedResult, result, 0.001);
}

TEST(DISABLED_SimpsonMethod, boost_test) {
    std::vector<double> leftsBorders = { 2, -1, 0, -3, 1 };
    std::vector<double> rightBorders = { 4, 5, 3, 1, 2 };
    std::function<double(std::vector<double>)> funcFiveVariable =
        [](std::vector<double> args) { return args[0] * args[1] * args[2] * args[3] * args[4]; };
    sizeType dimension = 5;
    std::vector<sizeType> countParts(5, 7);

    auto startSeq = tbb::tick_count::now();
    simpsonMethodSeq(dimension, leftsBorders, rightBorders, funcFiveVariable, countParts);
    auto finishSeq = tbb::tick_count::now();

    auto startParallel = tbb::tick_count::now();
    simpsonMethodParallel(dimension, leftsBorders, rightBorders, funcFiveVariable, countParts);
    auto finishParallel = tbb::tick_count::now();

    auto seqTime = (finishSeq - startSeq).seconds();
    auto parallelTime = (finishParallel - startParallel).seconds();
    auto boost = seqTime / parallelTime;

    printf("seqTime = %lf\nparallelTime = %lf\nboost = %lf\n", seqTime, parallelTime, boost);
}

\end{lstlisting}

\textbf{std::thread version}
\newline
\newline simpson\_method.h
\begin{lstlisting}
// Copyright 2022 Zarubin Mikhail

#ifndef MODULES_TASK_4_ZARUBIN_M_SIMPSON_METHOD_SIMPSON_METHOD_H_
#define MODULES_TASK_4_ZARUBIN_M_SIMPSON_METHOD_SIMPSON_METHOD_H_

#include <functional>
#include <vector>

const double ERROR = 0.001;

using sizeType = std::vector<double>::size_type;


double simpsonMethodSeq(sizeType dimension,
    std::vector<double> leftBorders, std::vector<double> rightBorders,
    std::function<double(std::vector<double>)> function, std::vector<sizeType> countParts);

double simpsonMethodParallel(sizeType dimension,
    std::vector<double> leftBorders, std::vector<double> rightBorders,
    std::function<double(std::vector<double>)> function, std::vector<sizeType> countParts,
    sizeType threadsCount = 12);

#endif  // MODULES_TASK_4_ZARUBIN_M_SIMPSON_METHOD_SIMPSON_METHOD_H_

\end{lstlisting}
simpson\_method.cpp
\begin{lstlisting}
// Copyright 2022 Zarubin Mikhail

#include <random>
#include "../../../3rdparty/unapproved/unapproved.h"
#include "../../../modules/task_4/zarubin_m_simpson_method/simpson_method.h"

struct Partition {
    double left;
    double center;
    double right;

    Partition() = default;
    Partition(double _left, double _center, double _right) :
        left(_left),
        center(_center),
        right(_right)
    {};
};

double simpsonMethodSeq(sizeType dimension,
    std::vector<double> leftBorders, std::vector<double> rightBorders,
    std::function<double(std::vector<double>)> function, std::vector<sizeType> countParts) {
    std::vector<double> coeffs(dimension);
    sizeType iterCount = 1;

    for (sizeType i = 0; i < dimension; i++) {
        coeffs[i] = (rightBorders[i] - leftBorders[i]) / countParts[i];
        iterCount *= countParts[i];
    }

    double integralValue = 0.0;
    sizeType count = static_cast<sizeType>(std::pow(6, dimension));
    std::vector<Partition> partitions(dimension);
    std::vector<double> args(dimension);

    for (sizeType i = 0; i < iterCount; i++) {
        sizeType temp = i;
        for (sizeType j = 0; j < dimension; j++) {
            double left = leftBorders[j] + temp % countParts[j] * coeffs[j];
            double right = leftBorders[j] + temp % countParts[j] * coeffs[j] + coeffs[j];
            double center = (right + left) / 2;

            partitions[j] = Partition(left, center, right);

            temp /= countParts[j];
        }

        for (sizeType k = 0; k < count; k++) {
            temp = k;

            for (sizeType j = 0; j < dimension; j++) {
                switch (temp % 6) {
                case 1:
                    args[j] = partitions[j].right; break;
                case 5:
                    args[j] = partitions[j].left; break;
                default:
                    args[j] = partitions[j].center; break;
                }

                temp /= 6;
            }

            integralValue += function(args);
        }
    }

    for (sizeType i = 0; i < dimension; i++) {
        integralValue *= (coeffs[i] / 6);
    }

    return integralValue;
}

void parallelSection(sizeType dimension, const std::vector<sizeType>& countParts,
    const std::vector<double>& leftBorders, const std::vector<double>& rightBorders,
    const std::vector<double>& coeffs, std::function<double(std::vector<double>)> function,
    sizeType begin, sizeType end, double* returnableValue) {
    double integralValue = 0.0;
    sizeType count = static_cast<sizeType>(std::pow(6, dimension));
    std::vector<Partition> partitions(dimension);
    std::vector<double> args(dimension);
    for (sizeType i = begin; i < end; i++) {
        sizeType temp = i;
        for (sizeType j = 0; j < dimension; j++) {
            double left = leftBorders[j] + temp % countParts[j] * coeffs[j];
            double right = leftBorders[j] + temp % countParts[j] * coeffs[j] + coeffs[j];
            double center = (right + left) / 2;

            partitions[j] = Partition(left, center, right);

            temp /= countParts[j];
        }

        for (sizeType k = 0; k < count; k++) {
            temp = k;

            for (sizeType j = 0; j < dimension; j++) {
                switch (temp % 6) {
                case 1:
                    args[j] = partitions[j].right; break;
                case 5:
                    args[j] = partitions[j].left; break;
                default:
                    args[j] = partitions[j].center; break;
                }

                temp /= 6;
            }

            integralValue += function(args);
        }
    }

    for (sizeType i = 0; i < dimension; i++) {
        integralValue *= (coeffs[i] / 6);
    }

    *returnableValue = integralValue;
}

double simpsonMethodParallel(sizeType dimension,
    std::vector<double> leftBorders, std::vector<double> rightBorders,
    std::function<double(std::vector<double>)> function, std::vector<sizeType> countParts,
    sizeType threadsCount) {
    std::vector<double> coeffs(dimension);
    sizeType iterCount = 1;

    for (sizeType i = 0; i < dimension; i++) {
        coeffs[i] = (rightBorders[i] - leftBorders[i]) / countParts[i];
        iterCount *= countParts[i];
    }

    std::vector<double> localIntegralValue(threadsCount, 0);
    std::vector<std::thread> threads(threadsCount);
    sizeType begin = 0, end = 0;
    for (sizeType i = 0; i < threadsCount; i++) {
        begin = end;
        end = (iterCount / threadsCount) * (i + 1) + (i == threadsCount - 1 ? iterCount % threadsCount : 0);

        threads[i] = std::thread(parallelSection, dimension, countParts,
            leftBorders, rightBorders,
            coeffs, function,
            begin, end, &localIntegralValue[i]);
    }

    double integralValue = 0.0;
    for (sizeType i = 0; i < threadsCount; i++) {
        threads[i].join();
        integralValue += localIntegralValue[i];
    }

    return integralValue;
}

\end{lstlisting}
main.cpp
\begin{lstlisting}

#include <gtest/gtest.h>
#include "./simpson_method.h"


TEST(SimpsonMethod, can_run_simpson_method_parallel) {
    std::vector<double> leftsBorders = { 1, 2 };
    std::vector<double> rightBorders = { 11, 9 };
    std::function<double(std::vector<double>)> func =
        [](std::vector<double> args) { return args[0] * args[1]; };
    sizeType dimension = 2;
    std::vector<sizeType> countParts = { 8, 4 };

    ASSERT_NO_THROW(simpsonMethodParallel(dimension, leftsBorders, rightBorders,
        func, countParts));
}

TEST(SimpsonMethod, simpos_method_parallel_works_correctly_with_one_variable) {
    std::vector<double> leftsBorders = { 2 };
    std::vector<double> rightBorders = { 4 };
    std::function<double(std::vector<double>)> funcOneVariable =
        [](std::vector<double> args) { return args[0]; };
    sizeType dimension = 1;
    std::vector<sizeType> countParts = { 10 };

    double expectedResult = 6.0;
    double result = simpsonMethodParallel(dimension, leftsBorders, rightBorders,
        funcOneVariable, countParts);

    ASSERT_NEAR(expectedResult, result, ERROR);
}

TEST(SimpsonMethod, simpos_method_parallel_works_correctly_with_two_variable) {
    std::vector<double> leftsBorders = { 2, -1 };
    std::vector<double> rightBorders = { 4, 5 };
    std::function<double(std::vector<double>)> funcTwoVariable =
        [](std::vector<double> args) { return args[0] * args[1]; };
    sizeType dimension = 2;
    std::vector<sizeType> countParts = { 5, 5 };

    double expectedResult = 72.0;
    double result = simpsonMethodParallel(dimension, leftsBorders, rightBorders,
        funcTwoVariable, countParts);

    ASSERT_NEAR(expectedResult, result, ERROR);
}

TEST(SimpsonMethod, simpos_method_parallel_works_correctly_with_three_variable) {
    std::vector<double> leftsBorders = { 2, -1, 0 };
    std::vector<double> rightBorders = { 4, 5, 3 };
    std::function<double(std::vector<double>)> funcThreeVariable =
        [](std::vector<double> args) { return args[0] * args[1] * args[2]; };
    sizeType dimension = 3;
    std::vector<sizeType> countParts = { 4, 4, 4 };

    double expectedResult = 324.0;
    double result = simpsonMethodParallel(dimension, leftsBorders, rightBorders,
        funcThreeVariable, countParts);

    ASSERT_NEAR(expectedResult, result, ERROR);
}

TEST(SimpsonMethod, simpos_method_parallel_works_correctly_with_four_variable) {
    std::vector<double> leftsBorders = { 2, -1, 0, -3 };
    std::vector<double> rightBorders = { 4, 5, 3, 1 };
    std::function<double(std::vector<double>)> funcFourVariable =
        [](std::vector<double> args) { return args[0] * args[1] * args[2] * args[3]; };
    sizeType dimension = 4;
    std::vector<sizeType> countParts = { 3, 3, 3, 3 };

    double expectedResult = -1296.0;
    double result = simpsonMethodParallel(dimension, leftsBorders, rightBorders,
        funcFourVariable, countParts);

    ASSERT_NEAR(expectedResult, result, ERROR);
}

TEST(SimpsonMethod, simpos_method_parallel_works_correctly_with_five_variable) {
    std::vector<double> leftsBorders = { 2, -1, 0, -3, 1 };
    std::vector<double> rightBorders = { 4, 5, 3, 1, 2 };
    std::function<double(std::vector<double>)> funcFiveVariable =
        [](std::vector<double> args) { return args[0] * args[1] * args[2] * args[3] * args[4]; };
    sizeType dimension = 5;
    std::vector<sizeType> countParts = { 2, 2, 2, 2, 2 };

    double expectedResult = -1944.0;
    double result = simpsonMethodParallel(dimension, leftsBorders, rightBorders,
        funcFiveVariable, countParts);

    ASSERT_NEAR(expectedResult, result, ERROR);
}

TEST(DISABLED_SimpsonMethod, boost_test) {
    std::vector<double> leftsBorders = { 2, -1, 0, -3, 1 };
    std::vector<double> rightBorders = { 4, 5, 3, 1, 2 };
    std::function<double(std::vector<double>)> funcFiveVariable =
        [](std::vector<double> args) { return args[0] * args[1] * args[2] * args[3] * args[4]; };
    sizeType dimension = 5;
    std::vector<sizeType> countParts(5, 6);

    auto startSeq = clock();
    simpsonMethodSeq(dimension, leftsBorders, rightBorders, funcFiveVariable, countParts);
    auto finishSeq = clock();

    auto startParallel = clock();
    simpsonMethodParallel(dimension, leftsBorders, rightBorders, funcFiveVariable, countParts);
    auto finishParallel = clock();

    auto seqTime = static_cast<double>(finishSeq - startSeq) / CLOCKS_PER_SEC;
    auto parallelTime = static_cast<double>(finishParallel - startParallel) / CLOCKS_PER_SEC;
    auto boost = seqTime / parallelTime;

    printf("seqTime = %lf\nparallelTime = %lf\nboost = %lf\n", seqTime, parallelTime, boost);
}

\end{lstlisting}
\end{document}